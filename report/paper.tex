\documentclass[sigplan,preprint,10pt]{acmart}\settopmatter{printfolios=true,printccs=false,printacmref=false}
\settopmatter{printacmref=false} 
\renewcommand\footnotetextcopyrightpermission[1]{}
\pagestyle{plain} 

\usepackage{lineno,hyperref,xcolor}
\usepackage{flushend}
\usepackage{stmaryrd}
\usepackage{amssymb}
\usepackage{xypic}
\usepackage{semantic}
\usepackage{booktabs} 
\usepackage{subcaption}
\usepackage{enumitem}
\usepackage[shortlabels]{enumerate}
\setlist{leftmargin=6mm}

\newcounter{thc}
\newcounter{dfc}

\theoremstyle{plain}
\newtheorem{lem}[thc]{Lemma}
\newtheorem{theorem}[thc]{Theorem}

\theoremstyle{definition}
\newtheorem{definition}[dfc]{Definition}

%\acmConference[PL'17]{ACM SIGPLAN Conference on Programming Languages}{January 01--03, 2017}{New York, NY, USA}
%\acmYear{2017}
%\acmISBN{} % \acmISBN{978-x-xxxx-xxxx-x/YY/MM}
%\acmDOI{} % \acmDOI{10.1145/nnnnnnn.nnnnnnn}
\startPage{1}
\setcopyright{none}
\bibliographystyle{ACM-Reference-Format}

\title{Wrattler: \textnormal{A platform for AI-assisted data science}}

\author{Authors}
%\affiliation{
%  \institution{The Alan Turing Institute}
%  \country{London, United Kingdom}
%}
%\email{tomas@tomasp.net}


\definecolor{cmtclr}{rgb}{0.0,0.6,0.0}
\definecolor{kvdclr}{rgb}{0.0,0.0,0.6}
\definecolor{numclr}{rgb}{0.0,0.4,0.0}
\definecolor{strclr}{rgb}{0.4,0.4,0.0}
\definecolor{rstrclr}{rgb}{0.5,0.1,0.0}
\definecolor{prepclr}{rgb}{0.6,0.0,0.2}
\newcommand{\vect}[1]{\langl #1 \rangl}
\newcommand{\langl}{\begin{picture}(4.5,7)
\put(1.1,2.5){\rotatebox{60}{\line(1,0){5.5}}}
\put(1.1,2.5){\rotatebox{300}{\line(1,0){5.5}}}
\end{picture}}
\newcommand{\rangl}{\begin{picture}(4.5,7)
\put(.9,2.5){\rotatebox{120}{\line(1,0){5.5}}}
\put(.9,2.5){\rotatebox{240}{\line(1,0){5.5}}}
\end{picture}}
\newcommand{\ball}[1]{\FPeval{\result}{clip(201+#1)}\textnormal{\ding{\result}}}
\newcommand{\lsep}{~\,|\,~}
\newcommand{\num}[1]{\textcolor{numclr}{#1}}
\newcommand{\str}[1]{\textnormal{\textcolor{strclr}{\sffamily "#1"}}}
\newcommand{\rstr}[1]{\textnormal{\textcolor{rstrclr}{\sffamily "#1"}}}
\newcommand{\ident}[1]{\textnormal{\sffamily #1}}
\newcommand{\qident}[1]{\textnormal{\sffamily \guillemotleft #1\guillemotright}}
\newcommand{\dom}{\ident{dom}}
\newcommand{\kvd}[1]{\textnormal{\textcolor{kvdclr}{\sffamily #1}}}

\newcommand{\bndclr}[1]{\textcolor{kvdclr}{#1}}
\newcommand{\bkndclr}[1]{\textcolor{prepclr}{#1}}
\newcommand{\blblclr}[1]{\textcolor{numclr}{#1}}
\newcommand{\bnd}[1]{\textnormal{\textcolor{kvdclr}{\sffamily #1}}}
\newcommand{\bknd}[1]{\textnormal{\textcolor{prepclr}{\sffamily #1}}}
\newcommand{\blbl}[1]{\textnormal{\textcolor{numclr}{\sffamily #1}}}


\begin{document}
\maketitle

\section{Introduction}
Data science is an iterative, exploratory process that requires a collaboration between a 
computer system and a human. A computer can provide advice based on statistical analysis of the 
data and discover hidden structures or corner cases, but only a human can decide what those mean
and decide how to handle them in data science scripts. Data science is often cited as an expensive
and time consuming task, especially due the costs of data cleaning and data wrangling.
We propose four fundamental reasons why
practical data science is expensive:

\paragraph{Big data is big,} 
so the analyst doesn't understand it all.*
Even if a data set is small enough fits on one computer, it's still too large to fit in 
an analyst's working memory.
But this means every analysis is huanted
by the spectre of lurking, potentially
unknown data quality  issues.
This also makes it more difficult to do data fusion, because there may be corner cases that make it more difficult to join two disparate data sources than expected.

\paragraph{The double Anna Karenina principle.}
Not only is every dirty data set  dirty in its own way, but \emph{pace} Tolstoy, every clean data set is clean in its own way as well. 
"Data" is such an abstract concept that specific integrity conditions to characterize whether data is dirty, and potentially even the most appropriate formalism for integrity conditions,  differ dramatically across
the vast array of disparate use cases of data science,
ranging form relational data describing the customers of a country, time series data describing sensor data in an internet of things platform, and huge datasets of satellite imagery of the earth over a multi-year time scale.

\paragraph{Death by a thousand cuts.} 
Often data transformation and processing steps are individually  very simple. But  there may need to be lots of them, and
because big data is big,
an analyst never knows if she has found them all.

\paragraph{Feedback cycles everywhere.} Data science is 
not a pipeline but a connected mess of epicycles. This is because every step 
in a data analysis actually teaches 
the analyst more about the data and the problem, which might require rehtinking the earlier steps. 
For example, there might be data quality issues that 
are not uncovered until the analyst
investigates the output of a regression model. 

\vspace{1em}
To meet these challenges, we present Wrattler,
a new type of system for data science
that aims to transform the process
of data analysis.
Wrattler combines the interactive and literal programming paradigms
of notebook systems such as Jupyter
with new advances in AI systems for data wrangling and in provenance.
The main design principles in Wrattler are:

\paragraph{Interactive.} Interactivity
is a necessary because ``big data is big'', so the analyst learns about 
the data set and the problem as she explores it.
Wrattler enables an efficient interaction by bringing computation closer to the human.
Notebooks run in the browser, cache partial results of computations and provide previews
of script results on-the-fly during development.

\paragraph{Reproducible.} 
Data analyses must be reproducible
because of feedback cycles. As the
analyst learns more about the problem,
this may uncover data cleaning or preparation issues that require
redesigning and rerunning the analysis.
Wrattler separates the task of running scripts from the task of managing state. 
This is handled by a data store, which tracks the provenance and semantics of data, supports 
versioning and keeps the history, making the data analyses fully reproducible.

\paragraph{Polyglot.}
Modern data science naturally draws
on many competing languages and libraries, such as R and Scipy. As a side effect
of our interactive, reproducible design,
we obtain nearly for free the ability
to support polyglot data analyses.
Multiple languages can be used in a single notebook and share data via the data store.
Analysts can use R and Python, but also interactive languages for data exploration
that run in the browser and provide live previews.

\paragraph{Smart.}
AI can examine and find patterns
in big data where a human cannot,
again aiming at the problem that
big data is big.
AI can be used to direct the analyst's attention and to generalize  decisions about data transformation
to new data that that the analyst
hasn't seen.
Wrattler serves as a platform for AI assistants that use machine learning to provide suggestions 
about data. Such AI assistants connect to the data store to infer types and meaning of data, provide 
help with data cleaning and joining, but also help data exploration by finding typical and atypical 
data points and automatically visualizing data.

\paragraph{Explainable.}
The hints provided by AI assistants are explainable. Rather than working as black boxes that 
transform one dataset into another, AI assistants generate scripts in simple domain-specific
languages, that specify how the data should be transformed. Those scripts can be reviewed 
and modified by a human.

\vspace{0.5em}
\noindent
In the rest of this document, we discuss limitations of current notebook systems and how Wrattler
resolves them (Section~\ref{sec:overview}). We discuss how the individual components of Wrattler
work together (Section~\ref{sec:wrattler}) and then focus on two of them in detail --
we look at how Wrattler can be extended with AI assistants (Section~\ref{sec:ai}) and
how semantic information about data is managed by the data store (Section~\ref{sec:datastore}).


\section{Wrattler and notebooks}
\label{sec:overview}

Notebook systems such as Jupyter became a popular programming environment for data science, because 
they support gradual data exploration and provide a convenient way of interleaving code with 
comments and visualizations. However, notebooks suffer from a number of issues that hanper 
reproducibility and limit the possible interaction model.

Notebooks can be used in a way that breaks reproducibility. The state is maintained by a \emph{kernel} 
and running a code in a cell overwrites the current state. There is no record of how the current 
state was obtained and no way to rollback to a previous state. The fact that the state is 
maintained by the kernel means that it is hard to combine multiple programming languages and
other components such as AI assistants. Finally, notebooks provide a very limited interaction 
model. To see the effect of a code change, an entire cell and all subsequent cells need to be
manually reevalutated.

\begin{figure}
\vspace{-0.5em}
\includegraphics[scale=0.6]{diagram.pdf}
\vspace{-0.5em}
\caption{\small{In notebook systems such as Jupyter, state and execution is managed by a kernel. In
  Wrattler, those functions are separated and enriched with AI assistants.}}
\label{fig:arch}
\vspace{-0.5em}
\end{figure}


The architecture of Wrattler allows us to address these issues, as well as to provide a platform
for building novel AI assistants and interactive programming. The architecture is illustrated
in Figure~\ref{fig:arch}. The components of Wrattler are:
%
\paragraph{Data store.} Imported external data, results of running scripts and of 
applying AI assistants are stored in the data store. It stores versioned data frames with 
metadata such as types, inferred semantics, data formats or provenance.

\paragraph{Language runtimes.} Scripts are evaluated by one or more language runtimes.
The runtimes read input data from and write results back to the data store.

\paragraph{AI assistants.} When invoked from the notebook, AI assistants read data
from data store and provide hints to the data analyst. They help to write data cleaning
scripts or annotate data in the data store with additional metadata such as inferred types.

\paragraph{Notebook.} The notebook is displayed in a web browser and orchestrates 
all other components. The browser builds a dependency graph between cells or individual 
expressions in the cells. It calls language runtimes to evaluate code that has changed,
AI assistants to provide hints and reads data from the data store to display results.  

\section{Wrattler components}
\label{sec:wrattler}

Wrattler consists of a notebook user interface running in the web browser, which communicates with 
a number of server-side components including language runtimes, the data store and AI assistants. 
In this section, we discuss the components in more detail, starting with the dependency graph
that is maintained on the client-side, by the web browser.

\subsection{Dependency graph}

When opening a notebook, Wrattler parses the source of the notebook (consisting of text cells and 
code cells) and constructs a dependency graph, which serves as a runtime representation of the
notebook. 

The structure is ilustrated in Figure~\ref{fig:deps}. Top-level nodes (squares) represent
individual notebook cells. Code (circles) is represented either as a single node (R, Python) or as 
a sub-graph with node for each operation (DSLs understood by Wrattler). Any node of a cell can 
depend on a data frame (hexagons) exported by an earlier cell. 
When code in a cell changes, Wrattler updates the dependency graph, keeping existing
nodes for entities that were unaffected by the change. 

The dependency graph enables several features that are difficult for most notebook systems:
%
\begin{itemize}
\item[--] When code changes, Wrattler only needs to recompute small part of the graph.
  This makes it possible to provide live previews, especially for simple data analytical DSLs.
\vspace{-0.85em}
\item[--] Refactoring can extract parts of the notebook that
  are needed to compute a specified output.
\vspace{0.25em}
\item[--] Refactoring can translate nodes of an analytical DSLs (generated by an AI assistant) into R or Python.
\vspace{0.25em}
\item[--] The graph can be used for other analyses and refactorings, such as provenance tracking
  or code cleanup.
\end{itemize}

\subsection{Data store}

The data store provides a way for persistently stroing data and enables communication between 
individual Wrattler components. Data store keeps external data files imported into Wrattler,
as well as computed data frames (and, potentially, other data structures). It is immutable 
and keeps past versions of data to allow efficient state rollback. For persistency and versioning, 
data store also serializes and stores multiple versions of the dependency graph.

The data store supports a mechanism for annotating data frames with additional semantic information, such as:
%
\begin{itemize}
\item[--] Columns can be annotated with (and stored as) primitive data type such as date or floating-point number.
\vspace{0.25em}
\item[--] Columns (or a combination) can be annotated with semantic annotation such as geo-location or and address.
\vspace{-0.85em}
\item[--] Columns, rows and cells of the data frame can be annotated with other metadata such as provenance.
\end{itemize}

\subsection{Language runtimes}

Language runtimes are responsible for evaluating code and assisting with code analysis during
the construction of dependency graph. They can run as server-side components (e.g.~for R and Python) 
or as client-side components (for JavaScript and small analytical DSLs used by AI assistants). 
Unlike Jupyter kernels, a language runtime is stateless. It reads data from and writes data to
the data store. (Using a cache for efficiency.)


\begin{figure}
\includegraphics[scale=1,trim=0.5cm 0.5cm 0.5cm 0.5cm]{graph.pdf}

\caption{\small{Sample dependency graph. Data acquisiton and analysis are represented as 
opaque R/Python code cells. Data cleaning is written in a domain-specific langauge understood by
Wrattler and represented by multiple cells. Latter two cells depend on data frames exported by
earlier cells.}}
\label{fig:deps}
\vspace{-0.5em}
\end{figure}

\subsection{AI assistants}
\label{sec:wrattler-ai}

The Wrattler architecture enables integration of AI assistants in a number of ways. First, AI
assistants have access to the data store (and the dependency graph) and can use those to provide
guidance about both data and code. Second, Wrattler includes an extensible framework for defining
domain-specific languages (DSLs) that can describe, for example, data extraction, cleaning, transofrmation or 
visualization. An AI assistant can work in two ways:
%
\begin{itemize}
\item[--] When invoked, the assistant creates a new frame in the data store with additional
  information, e.g. probabilistic type inference (ptype) will annotate columns (and possibly
  also cells) with their probabilistic types.
\vspace{0.25em}
\item[--] The assistant defines a DSL for the task at hand. When invoked from a notebook,
  it guides the user in creating a script in the DSL that performs the required operation
  (such as extracting data or reformatting a data frame).
\end{itemize}
%
The second case is illustrated in Figure~\ref{fig:arch}. An AI assistant reads data from the data
store and returns suggestions in a small DSL to the notebook, which the user can review 
(using a live preview mechanism) and accept. For example, an AI
assistant based on datadiff can suggest the following script (written using the Wrattler DSL
framework):
%
\begin{equation*}
\begin{array}{l}
\ident{datadiff(broadband2014, broadband2015)}\\
\quad.\ident{drop\_column}(\str{WT\_national})\\
\quad.\ident{drop\_column}(\str{WT\_ISP})\\
\quad.\ident{recode\_column}(\str{URBAN}, [\num{1}, \num{2}], [\str{Urban},\str{Rural}])
\end{array}  
\end{equation*}
%
The script specifies that two columns should be dropped from the badly structured data frame and
one column needs to be recoded (turning $\num{1}$ and $\num{2}$ into strings
\strf{Urban} and \strf{Rural}).

The above script is constructed interactively. When the user types the first line and types
``.'' the datadiff assistant offers the most likely patches and the user can choose. While
choosing, a preview of the result is computed on-the-fly, giving the user an immediate feedback.
Finally, the script is represented in a fine-grained way (second cell in Figure~\ref{fig:deps}).
Once it is interactively constructed, it can be translated to multiple supported languages
such as R, Python or JavaScript.

\section{AI assistants}
\label{sec:ai}

{
An AI assistant is a component
of the system that 
guides data analysts through a single data collection, integration, preparation, 
analytics, or reporting task.
For example, an assistant might
specialize in identifying suspicious
values, such as "-1" in a column labelled
"heart rate", which might indicate
that certain data was not collected.
Assistants are interactive, in that
they propose a transformation
of the data to the analyst, based on combinations
of statistical and symbolic AI analysis,
and then refine the transformation
based on feedback from the analyst.
Typically analysts will interact with many separate assistants over 
the course of an analysis, each of which specialize in a single kind of task.
}
\subsection{Architecture of AI assistants}

There are two important design aspects of AI assistants, which may seem 
contradictory.
 First, they must be interactive, following
 the first principle behind Wrattler, but
 they must also be reproducible.
We square this circle by using the notebook
system as a reproducible interlingua. 
The output of an AI assistant is always code, rather
than being a black-box that transforms data,
and it is this code that is agreed between the AI and the analyst during the interaction.
Thus the interaction is viewed as a ``design phase''
that is not reproduced on new data.

Assistants are therefore based on two different ideas:

\paragraph{Domain specific languages.}
The output of an assistant is code in a small domain-specific language designed for the task that
the AI assistant is solving. The code can be used in two ways -- during the interaction with the
AI assistant, the code is exposed through a Wrattler DSL framework as discussed in 
Section~\ref{sec:wrattler-ai}. For production use, the code is translated to a language like R or 
Python. A typical code produced by an AI assistant will implement a data transformation or
produce a visualization.

\paragraph{Interactivity.}
Assistants should usually contain user interaction in their design, and adapt rapidly to feedback 
from the analyst. For example, a data cleaning assistant would probably want to ask the user to 
confirm potential changes, and if the user says no, to reconfigure the cleaning module 
significantly so that the user does not need to keep rejecting similar changes. 

However, at the same time, our data preparation tools should be usable in "batch mode" by analysts 
who already know what they want to do and don't want to mess with a wizard. We should aim to 
produce the equivalent of scikit-learn for data preparation.

The output snippet should be readable, and something that a data analyst can look at to verify 
that the transformation accomplishes what they had wanted. As far as possible, the snippet 
should be "reusable" in a sense that it could be copy-pasted into another script that was 
processing a similar dataset and would still work. You can think of a snippet as something 
that a person might write in a single cell of a Jupyter notebook.


\vspace{1em}
AI assistants have access to information in the Wrattler data store and can use two separate
aspects for their guidance:
%
\begin{itemize}
\item[--] AI assistants can access all data stored in the data
store. This includes imported external data files, such as raw text or raw HTML content from which
the analyst wants to extract data, as well as structured data frames imported from CSV files or
produced as a result of previous computations. Moreover, an AI assistant can also access multiple
versions of a file.
\vspace{0.25em}

\item[--] AI assistants can access the dependency graph created for a 
notebook. This allows the them to access the transformation history, which is the code 
that has already been run during the analysis. The idea is that assistants may need to know the 
context of transformations that have already been performed in order to decide what to suggest next.
\end{itemize}

The assistants should be self-contained in the sense that they interact with each other
only through the 
data store. This restriction is necessary in order
for provenance tracking and reproducibility.



\subsection{The Assistant Landscape}
{

AI assistants can be imagined for every step of the data analytics process. 
To give an idea of the breadth of this framework,
we present a broad list of the data analytics tasks which have potential for assistant automation.
This is an intentionally broad list, more so than
could be fully explored in even a large research project, but gives a sense of the breadth of 
potential use cases of our framework.

\paragraph{Data collection.} Assistants can search the data store, as well as external repositories
to recommend data sets that are relevant to the task at hand and offer to import these. For example,
when analysing data about different countries, 
the assistant could suggest joining in public
demographic information from DBPedia.

\paragraph{Data integration} is a problem as common
as it is difficult. We envisage assistants
that combine logical and statistical reasoning to assist
with data  integration. A simple but useful
example of a tool in this space is to
 identifying pairs of variables across multiple data 
sources that are likely to be daily temperatures.
More sophisticated examples could include and extend the
most cutting edge techniques in the data fusion and
semantic technologies literatures.

\paragraph{Schema inference.} A sad fact of data
science is that data sets are often not as well documented
as we might like, even to the minimal level of
documentation of the types of each column.
There is an interesting opportunity to
develop machine learning methods that inferring types of variables from their values, e.g.,
by noticing that integer values between 0 and 30 in 
the United Kingdom might describe an outdoor temperature, especially if other weather-related
variables are detected in the data set.

\paragraph{Data parsing.}
A common task in data analysis is to convert
data that is ``almost'' in structured form,
such as HTML tables, into a relational table.
Assistants in this space include
tools specifying a scraper for web pages, related
to wrapper induction \cite{Kushmerick1997WrapperIF},
for inferring format of CSV files (while simple
conceptually, they have no standard format), and
for extracting tables from html and pdf files
\cite{pinto03table}.

\paragraph{Data cleaning} is a well studied
problem in the databases and data mining literature
\cite{abedjan2016detecting,ilyas2015},
but there are still a huge number of issues that have been unexplored.
For example, we envisage assistants for
Identifying values that seem to be out of range, e.g. missing value indicators;
Identifying possible inconsistent coding of strings,
such as different representations of acrynoms;
indentifying rows that seem "jointly" unusual, i.e., that seem not to conform to statistical dependencies among the columns; and
identifying distinct records that may refer to the same entity (record linkage).

\paragraph{Data monitoring.} Once an analysis
is completed, it may result in a machine learning
model which will then be deployed on a data stream.
Once a classifier is deployed, it must be monitored 
to detect shifts in its input distribution,
such as change points, covariate shift, and
concept drift. This includes the problem 
to suggesting when/how often new data points should be labelled for monitoring.
Even for data analyses that do not involve
deploying a ``production model'', there is
often a need to run an analysis periodically on updated data, such as re-running last quarter's
analysis on this quarter's data.
In such cases, we might believe that the two data
sets come from approximately the same distribution
and format, so we can exploit
divergences from this assumption to indicate
potential issues with data quality.
As part of the Wrattler effort, we have
introduced the \emph{data diff} problem \cite{datadiff},
which aims to improve the process of data wrangling by 
returning a report of the differences between two data sets.

\paragraph{Exploratory visualization} to summarize
data sets, both the target data of the analysis,
and the results of transformations that the analyst
creates. Potential assistants here include ones
to summarize most representative rows of a data table,
and automatically choosing appropriate interactive
visualizations, such as scatterplots and bar charts,
with styling decisions such as axes automatically
chosen to be most informative.

\paragraph{Performing analytics.}
An increasing amount of attention is being
given to the AutoML challenge \cite{guyon_review_2016}, which focuses
on automatically choosing a classification
methods, feature representations and hyperparameters
for a given classification task, without the need for human intervention.

\paragraph{Debugging and interpreting predictive analytics.}
Every time a new machine learning model is trained,
the first question an analyst has is often
why its accuracy is not better. But
debugging predictive analytics is notoriously difficult, and the remedies indirect and expensive,
 e.g., labelling more data, adding more features,
 using a more complex model. Debugging and improving
 models is a rich area for potential assistants.
 There is potentially low hanging fruit available
 in how to aid the tasks of
error analysis, exploring the
predictions of the model 
on validation data. Some early work in this direction is
\cite{saleema}.
There is a rich and growing literature on
interpreting and explaining the predictions
of a model \cite{lipton:mythos,doshi-velez17,ribiero2016lime,darksight}.
Two potential directions from data are
debugging by using
"training set blame": Given an error in the validation data, what examples from the training set caused me to make that error? This might build on the work of \cite{percy}.
Another potential avenue is explaining
differences between predictions: why was instance 42 treated differently than instance 24601, even though they appear highly similar to a human?
}

\vspace*{1em}

%%cs possibly we don't need to go into detail
%% in this version of the report? Or we might
%% decide to add this back.
\begin{comment}
\subsection{Possible AI assistants}

Designing an AI assistant requires an intriguing
combination of data science, AI, and user interaction
thinking.
As an example of the types of design considerations
that arise, we go into more detail about the
design issues that arise when considering
three examples of the broader list of agents
in the previous section.

{\color{blue}

\paragraph{Data Parsing.}

Simple scraper assistant: Simple information extraction wizard. Create a data table based
on HTML pages that have been automatically generated, like product review pages,
for which a simple pattern on the HTML tree will be sufficient to extract the desired information.
This is called wrapper induction in the information extraction literature.

* *Input*: Web page that contains information in unstructured form, such as a list of search resuts or an HTML list.

* *Output*: A data frame (i.e. a relational table) that contains the information that has
been extracted from the web page.

* *User interaction*: User needs to initially highlight and specify which items to be scraped.
Need to summarize the results of the scraping and highlight cases
which might require correction from the user. Update scraper accordingly.

* *Potential method*: A wrapper is a path through the HTML DOM. Use a language (like XPath) for navigating through the DOM. Search for the shortest path that retrieves
the labels provided by the user.

* *Research issues*: Not clear, given the large amount of related work in this area.
We would need to read around carefully before we can identify opportunities.
Possible opportunities include:
    (a) Using a language like [Scrapy](https://scrapy.org/) to induce the wrappers;
    (b) Using a language model over the scraped values to fine-tune the wrapper (e.g. if a wrapper returns both dates and cities, it's probably wrong), or
	(c) using a probabilistic model over the program representation of the wrapper to focus attention in the search
for good wrappers. Perhaps there are enough scrapers on line (e.g. Github projects that use scrapy) that you can get a training set for this.
* *Related work*: Gulwani's work on data wrangling, lots of work on wrapper induction.
There is a lot of work on this. Scrapy is a popular open source framework that several
of my BSc/MSc students have found useful.

\paragraph{Data cleaning.}
*Inconsistent string assistant*: Identify potential typos in string data. The idea is to look for strings that do not occur commonly in the data set, but are very close in edit distance
to strings that do occur commonly.

* *Input*: A column of a data table whose values are strings, like addresses, cities, states/provinces, occupations, and so on. 

* *Output*: A list of suggestions for strings that should be renamed: e.g. `[ ("Stocland", "Scotland"), ("Untied Kringdom", "United Kingdom") ]`

* *Potential method*: Simplest cut: Define a threshold $\delta$ on Levenshtein distance,
and $0 < \alpha < 1$ on the number of occurrences. Return the set of all strings $y$ that
occur less than $\alpha N$ times in the data set but where there is a corrected string $x$ which is less than $\delta$ in string edit distance from $y$.
More ambitious: Noisy channel model for correction.
Learn a probability distribution $p(x)$ over the strings in the column,
and  define an error model $p(x'|x)$. Then search for ways to transform the  dirty column $x'$  to a clean version $x$ in such a way to maximize $p(x|x')$.
This formulation deserves some more thought,
though, before we try to implement it.

* *Research questions*: Need to connect this to existing concepts in the databases and
data mining literature, such as U-repair.

* *User Interaction*: The user should be able to reject suggestions in the output list,
and this should rapidly update the models used in such a way that similar suggestions
to the bad ones are also removed from the list. A noisy channel model would be nice
for this, not sure about the edit distance one.

\paragraph{Exploratory visualization.}

*Data table summarization assistant*: Displays a short subtable of the most "representative"
columns of a given data frame. The number of rows in the summary should be small enough
to fit on a given screen. Essentially an alternative to the `head()` function in pandas/R.

*Input*:  A data frame, along with type information (continuous/categorical) for each of the columns. Desired number $k$ of rows in output. 

*Output*: A list of $k$ indices of rows in the dataframe

*Potential method*:  Simplest possible implementation:
Run $k$-medoids and return the medoid values.

*Research questions:* First, are there research questions here, or is this so simple
that it  is part of the infrastructure?  If $k$-medoids is so simple for this,
why doesn't everyone also do it? How to evaluate whether the summary is good?
Maybe there are examples of data sets where medoids is not good.
For example, in the Karpathy ICML example, medoids would not make sense,
listing the institutions that have the most papers is most informative, because
it's a "long tail" type of column. How do you tell "long tail" data from "just show
the clusters" type data? Perhaps a model-based framework could distinguish?
Maybe you want to summarize the ways in which two data sets differ?
Is the best summary the cluster centroids or the "top k" along some value? 
* *User interaction:* I'm not sure what interactions are enecessary.
Users could ask for more rows, mark two items as similar, or mark an item as uninteresting.
Or perhaps better would be to allow drill down, i.e., to make it easy to explore
the clusters represented by each example.
}
\end{comment}

\section{Data store}
\label{sec:datastore}

{\color{red}
\textbf{TP: My rough notes from discussion with JG}

~

\noindent
Data store stores data frames with metadata. This is a good start for data science work,
because most languages can provide mappings to/from data frames.
}

\subsection{Semantics annotations}
{\color{red}
Data store provides a way of annotating cells, columns, combinations of columns and rows.
For example, a column may be annotated with a type that indicates that its values are a mix of
postcodes and city names. A cell can be annotated with an information saying that this value is 
more likely a postcode than a city name. We also need to support annotaitons on a combination of
columns, because three columns can jointly represent an address. A row can be annotated with a 
provenance -- for example, when merging rows from two data sets with different sources.
}

\subsection{Annotation format}
{\color{red}
We want a lightweight annotation format so that implementing an AI assistant or a language
runtime that needs to communicate with the data store is not too hard.
Our annotation format will follow something like \url{http://schema.org}, which defines
a hierarchy of entity types and has a simple way for adding annotations to things.
We will need to define our own hierarchy for schemas for data science.
}

\subsection{Schemas for data science}
{\color{red}
This hierarchy will include some basic things that the data store understands and can 
convert between (integers, floats, unbounded size integers, ISO dates, etc.)

In addition, we will define some purely semantic annotations that are commonly useful and 
we will define a hierarchy for those. This represents things like addresses, postcodes, countries,
cities, etc.

Everyone can define their own schemas (like schema.org) by defining their own namespace (URL)
and providing entities with their custom namespace. The data store will simply save and return
this information - without doing any checking.}

\subsection{Content negotiation}
{\color{red}
The only bit where the data store does something clever is that it will be able to return data
in a format that the client asked for (a bit like HTTP negotiation). The client can say, ``I only
understand semantic information defined by these schema namespaces`` and the data store will do
its best to return data in the required format.

For simple types (e.g. dates) that the data store understands, it will convert data accordingly.
For other semantic information that the data store does not understand (e.g. information that
a certain cell is missing with a probability $p$), the data store will filter out the extra
information and just return the raw number.
}

\bibliographystyle{plain}
\bibliography{paper}
\end{document}
